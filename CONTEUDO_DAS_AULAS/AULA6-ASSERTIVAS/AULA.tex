%%%%%%%%%%%%%%%%%%%%%%%%%%%%%%%%%%%%%%%%%
% fphw Assignment
% LaTeX Template
% Version 1.0 (27/04/2019)
%
% This template originates from:
% https://www.LaTeXTemplates.com
%
% Authors:
% Class by Felipe Portales-Oliva (f.portales.oliva@gmail.com) with template 
% content and modifications by Vel (vel@LaTeXTemplates.com)
%
% Template (this file) License:
% CC BY-NC-SA 3.0 (http://creativecommons.org/licenses/by-nc-sa/3.0/)
%
%%%%%%%%%%%%%%%%%%%%%%%%%%%%%%%%%%%%%%%%%

%----------------------------------------------------------------------------------------
%	PACKAGES AND OTHER DOCUMENT CONFIGURATIONS
%----------------------------------------------------------------------------------------

\documentclass[
	12pt, % Default font size, values between 10pt-12pt are allowed
	%letterpaper, % Uncomment for US letter paper size
	%spanish, % Uncomment for Spanish
]{fphw}

% Template-specific packages
\usepackage[utf8]{inputenc} % Required for inputting international characters
\usepackage[T1]{fontenc} % Output font encoding for international characters
\usepackage{mathpazo} % Use the Palatino font
\usepackage{setspace}

\usepackage{graphicx} % Required for including images

\usepackage{booktabs} % Required for better horizontal rules in tables

\usepackage{listings} % Required for insertion of code

\usepackage{enumerate} % To modify the enumerate environment

\usepackage{parskip}% http://ctan.org/pkg/parskip

%----------------------------------------------------------------------------------------
%	ASSIGNMENT INFORMATION
%----------------------------------------------------------------------------------------

\title{AULA 4 - Assertivas} % Assignment title

\author{AVC, JPP, PC} % Student name

\date{} % Due date

\institute{Pontifícia Universidade Católica do Rio de Janeiro \\ Departamento de Informática} % Institute or school name

\class{Programação Modular (INF1301)} % Course or class name

\professor{Flavio Bevilacqua} % Professor or teacher in charge of the assignment

%----------------------------------------------------------------------------------------

\begin{document}

\maketitle % Output the assignment title, created automatically using the information in the custom commands above

%----------------------------------------------------------------------------------------
%	ASSIGNMENT CONTENT
%----------------------------------------------------------------------------------------
\begin{doublespace}

Qualidade por construção: garantir a qualidade de tudo que é constuído ao longo do projeto.

\begin{enumerate}

    \item Definição
    
    Regras consideradas válidas em determinado ponto de execução.

    Assertivas são colocadas ao longo do código para nos certificarmos que o resultado do código até então é o que esperamos. $printf$ é uma forma de tratar assertivas.
    
    \item Onde são utilizadas

    \begin{itemize}

        \item Na busca pela origem do problema.
        \item Argumentação de corretude.
        \item Instrumentação (distribuir controles no código para que ele detecte erros).
        \item Pseudoinstruções (comentários colocados ao longo do código que especificam o que esperamos ser feito).
        \item Depuradores (utilização de breakpoints).
        
    \end{itemize}

    \item Assertivas de entrada e assertivas de saída
    
    AE: aquilo que deve ser verdadeiro antes da execução da função. AEs são escritas antes do código da função a qual elas se referem.

    AS: aquilo que deve ser verdadeiro depois da execução da função. ASs são escritas depois do código da função a qual elas se referem.

    \item Exemplo
    
    Exclusão do nó intermediário de uma lista duplamente encadeada com cabeça.

    AE:

    \begin{itemize}
        \item Ponteiro corrente aponta para o nó intermediário a ser excluído.
        \item Valem as assertivas estruturais da lista duplamente encadeada com cabeça.
    \end{itemize}

    AS:

    \begin{itemize}
        \item Ponteiro corrente aponta para o nó seguinte.
        \item O nó foi excluído.
        \item Valem as assertivas estruturais da lista duplamente encadeada com cabeça.
    \end{itemize}

\end{enumerate}

\end{doublespace}
    %----------------------------------------------------------------------------------------

\end{document}
