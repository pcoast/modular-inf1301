%%%%%%%%%%%%%%%%%%%%%%%%%%%%%%%%%%%%%%%%%
% fphw Assignment
% LaTeX Template
% Version 1.0 (27/04/2019)
%
% This template originates from:
% https://www.LaTeXTemplates.com
%
% Authors:
% Class by Felipe Portales-Oliva (f.portales.oliva@gmail.com) with template 
% content and modifications by Vel (vel@LaTeXTemplates.com)
%
% Template (this file) License:
% CC BY-NC-SA 3.0 (http://creativecommons.org/licenses/by-nc-sa/3.0/)
%
%%%%%%%%%%%%%%%%%%%%%%%%%%%%%%%%%%%%%%%%%

%----------------------------------------------------------------------------------------
%	PACKAGES AND OTHER DOCUMENT CONFIGURATIONS
%----------------------------------------------------------------------------------------

\documentclass[
	12pt, % Default font size, values between 10pt-12pt are allowed
	%letterpaper, % Uncomment for US letter paper size
	%spanish, % Uncomment for Spanish
]{fphw}

% Template-specific packages
\usepackage[utf8]{inputenc} % Required for inputting international characters
\usepackage[T1]{fontenc} % Output font encoding for international characters
\usepackage{mathpazo} % Use the Palatino font
\usepackage{setspace}

\usepackage{graphicx} % Required for including images

\usepackage{booktabs} % Required for better horizontal rules in tables

\usepackage{listings} % Required for insertion of code

\usepackage{enumerate} % To modify the enumerate environment

\usepackage{parskip}% http://ctan.org/pkg/parskip

\usepackage{amsmath}

\usepackage{amssymb}

\usepackage{caption}

\usepackage{tikz}
\newcommand*\circled[1]{\tikz[baseline=(char.base)]{
            \node[shape=circle,draw,inner sep=2pt] (char) {#1};}}

\usepackage{enumitem}

\usepackage[ruled,portuguese,onelanguage,longend]{algorithm2e} %for psuedo code}% http://ctan.org/pkg/algorithm2e
% \makeatletter
% \renewcommand{\@algocf@capt@plain}{above}% formerly {bottom}
% \makeatother

%----------------------------------------------------------------------------------------
%	ASSIGNMENT INFORMATION
%----------------------------------------------------------------------------------------

\title{AULA 9 - Argumentação de Corretude Continuação} % Assignment title

\author{AVC, JPP, PC} % Student name

\date{} % Due date

\institute{Pontifícia Universidade Católica do Rio de Janeiro \\ Departamento de Informática} % Institute or school name

\class{Programação Modular (INF1301)} % Course or class name

\professor{Flavio Bevilacqua} % Professor or teacher in charge of the assignment

%----------------------------------------------------------------------------------------

\begin{document}

\maketitle % Output the assignment title, created automatically using the information in the custom commands above

%----------------------------------------------------------------------------------------
%	ASSIGNMENT CONTENT
%----------------------------------------------------------------------------------------
\begin{doublespace}

    \begin{algorithm}[h]

        \caption{Insertion Sort}

        \SetAlgoLined
        % \KwData{this text}
        % \KwResult{how to write algorithm with \LaTeX2e }
        AE $\longrightarrow$

        \Indp\Inicio
        {

            IND $\longleftarrow$ 1

            \Enqto{IND $\leqslant$ LIMITE-LÓGICO}
            {

                ATUAL $\longleftarrow$ ELE[IND]

                AUX $\longleftarrow$ IND - 1

                \Enqto{AUX $>$ 0 E ELE[AUX] $>$ ATUAL}
                {
                    ELE[AUX + 1] $\longleftarrow$ ELE[AUX]

                    AUX $\longleftarrow$ AUX - 1
                }

                ELE[AUX + 1] $\longleftarrow$ ATUAL

                IND $\longleftarrow$ IND + 1

            }

        }

        \Indm AS $\longrightarrow$

    \end{algorithm}

    \textbf{Argumentação de Sequência 1}

    AE: Existe um vetor a ser ordenado.

    AS: Vetor está vazio ou foi ordenado.

    AI 1: IND aponta para o primeiro elemento do vetor.

    \textbf{Argumentação de Repetição 1}

    AE: AI 1.

    AS: AS geral.

    AINV:

    \begin{itemize}

        \item Existem dois conjuntos: a ordenar e já ordenados.
        \item IND aponta para o elemento a ordenar.
        
    \end{itemize}

    \begin{enumerate}[label=\protect\circled{\arabic*}]
        \item AE $\Longrightarrow$ AINV
        
        \begin{itemize}
            \item Pela AE, IND aponta para o primeiro elemento do vetor. Todos os elementos estão no conjunto a ordenar e o conjunto já ordenados está vazio. Logo, vale a AINV.
        \end{itemize}

        \item AE \&\& (Condição == False) $\Longrightarrow$ AS

        \begin{itemize}
            \item Pela AE, IND == 1. Para que (Condição == False), LIMITE-LÓGICO = 0, ou seja, vetor está vazio. Neste caso, vale a AS.
        \end{itemize}

        \item AE \&\& (Condição == True) \circled{+} B $\Longrightarrow$ AINV

        \begin{itemize}
            \item Pela AE, IND aponta para o primeiro elemento do vetor que não está vazio. Este elemento já se encontra no local ordenado. Com isso, os dois conjuntos existem e um elemento de a ordenar passou para já ordenados. Logo, vale a AINV.
        \end{itemize}

        \item AINV \&\& (Condição == True) \circled{+} B $\Longrightarrow$ AINV

        \begin{itemize}
            \item Para que a AINV continue valendo, B deve garantir que um elemento passe do conjunto a ordenar para já ordenado e IND seja reposicionado.
        \end{itemize}

        \item AINV \&\& (Condição == False) \circled{+} B $\Longrightarrow$ AS

        \begin{itemize}
            \item Se (Condição == False), IND ultrapassou o LIMITE-LÓGICO, ou seja, todos os elementos passaram para o conjunto já ordenados. Como o vetor está ordenado, vale a AS.
        \end{itemize}

        \item Término
        
        \begin{itemize}
            \item Como a cada ciclo, um elemento é retirado do conjunto a ordenar, e este conjunto possui um número finito de elementos, a repetição terminará em um número finito de passos.
        \end{itemize}

    \end{enumerate}

    \pagebreak

    \textbf{Argumentação de Sequência 2}

    AE (seq2) = AS (seq2) = AINV

    AI 2: ATUAL recebe elemento a ser reposicionado.

    AI 3: AUX aponta para o último elemento do conjunto já ordenado (caso não esteja vazio) ou AUX $\longleftarrow$ 0.

    AI 4: Local para onde ATUAL será reposicionado foi definido.

    AI 5: ATUAL foi reposicionado.
    
    OBS: o último bloco (IND $\longleftarrow$ IND + 1) garante que IND foi reposionado para o próximo elemento a ordenar ou ultrapassou o LIMITE-LÓGICO. Vale a AINV ou a AS geral se o vetor está ordenado.

    \textbf{Argumentação de Repetição 2}

    AE: AI 3.

    AS: AI 4.

    AINV:

    \begin{itemize}

        \item Existem dois conjuntos: maiores e possíveis menores.
        \item AUX aponta para elemento de possíveis menores.
        
    \end{itemize}

    \begin{enumerate}[label=\protect\circled{\arabic*}]
        \item AE $\Longrightarrow$ AINV
        
        \begin{itemize}
            \item Pela AE, AUX aponta para o último elemento de possíveis menores e o conjunto de maiores está vazio. Vale a AINV.
        \end{itemize}

        \item AE \&\& (Condição == False) $\Longrightarrow$ AS

        \begin{itemize}
            \item Para que (Condição == False) antes do primeiro ciclo da repetição, AUX == 0 e IND == 1. IND aponta para o primeiro elemento a ser reposicionado e ele já se encontra posicionado, valendo a AS.
        \end{itemize}

        \item AE \&\& (Condição == True) \circled{+} B $\Longrightarrow$ AINV

        \begin{itemize}
            \item Como o primeiro ciclo executou, o elemento apontado por AUX é maior do que ATUAL. Com isso, ele passou do conjunto de possíveis menores para maiores e AUX foi reposicionado. Vale a AINV.
        \end{itemize}

        \item AINV \&\& (Condição == True) \circled{+} B $\Longrightarrow$ AINV

        \begin{itemize}
            \item Para que a AINV continue valendo, B deve garantir que um elemento passe do conjunto a possíveis menores para maiores e AUX seja reposicionado.
        \end{itemize}

        \item AINV \&\& (Condição == False) \circled{+} B $\Longrightarrow$ AS

        \begin{itemize}
            \item Se (Condição == False), AUX == 0, ou seja, todos os elementos de possíveis foram reposicionados no conjunto maiores. Isso significa que ATUAL é menor do que todos. Logo, a posição em que ele será reposicionado é AUX == 0 + 1 == 1. Vale a AS.
            \item Se (Condição == False), ELE[AUX] $\longleftarrow$ ATUAL, se ocorrer este teste, todos os elementos à esquerda de AUX serão menores que ATUAL. Com isso, foi definido o local em que ATUAL será reposicionado, valendo a AS.
        \end{itemize}

        \item Término
        
        \begin{itemize}
            \item Como o número de elementos do conjunto possíveis menores é finito e cada ciclo retira um elemento deste conjunto, a repetição terminará em um número finito de passos.
        \end{itemize}

    \end{enumerate}

\end{doublespace}
%----------------------------------------------------------------------------------------

\end{document}
