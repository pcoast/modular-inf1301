%%%%%%%%%%%%%%%%%%%%%%%%%%%%%%%%%%%%%%%%%
% fphw Assignment
% LaTeX Template
% Version 1.0 (27/04/2019)
%
% This template originates from:
% https://www.LaTeXTemplates.com
%
% Authors:
% Class by Felipe Portales-Oliva (f.portales.oliva@gmail.com) with template 
% content and modifications by Vel (vel@LaTeXTemplates.com)
%
% Template (this file) License:
% CC BY-NC-SA 3.0 (http://creativecommons.org/licenses/by-nc-sa/3.0/)
%
%%%%%%%%%%%%%%%%%%%%%%%%%%%%%%%%%%%%%%%%%

%----------------------------------------------------------------------------------------
%	PACKAGES AND OTHER DOCUMENT CONFIGURATIONS
%----------------------------------------------------------------------------------------

\documentclass[
	12pt, % Default font size, values between 10pt-12pt are allowed
	%letterpaper, % Uncomment for US letter paper size
	%spanish, % Uncomment for Spanish
]{fphw}

% Template-specific packages
\usepackage[utf8]{inputenc} % Required for inputting international characters
\usepackage[T1]{fontenc} % Output font encoding for international characters
\usepackage{mathpazo} % Use the Palatino font
\usepackage{setspace}

\usepackage{graphicx} % Required for including images

\usepackage{booktabs} % Required for better horizontal rules in tables

\usepackage{listings} % Required for insertion of code

\usepackage{enumerate} % To modify the enumerate environment
\usepackage{caption}

%----------------------------------------------------------------------------------------
%	ASSIGNMENT INFORMATION
%----------------------------------------------------------------------------------------

\title{AULA 3 - Principios da modularidade parte 3} % Assignment title

\author{AVC, JPP, PC} % Student name

\date{} % Due date

\institute{Pontifícia Universidade Católica do Rio de Janeiro \\ Departamento de Informática} % Institute or school name

\class{Programação Modular (INF1301)} % Course or class name

\professor{Flavio Bevilacqua} % Professor or teacher in charge of the assignment

%----------------------------------------------------------------------------------------

\begin{document}

\maketitle % Output the assignment title, created automatically using the information in the custom commands above

%----------------------------------------------------------------------------------------
%	ASSIGNMENT CONTENT
%----------------------------------------------------------------------------------------
\begin{doublespace}

    \begin{enumerate}

        \item Tipo Abstrato de Dados (TAD)

              TAD é uma estrutura encapsulada que só é conhecida externamente pelas funções de acesso que a manipulam.

              A Estrutura Encapsulada da Programação Modular equivale aos Atributos da Programação Orientada a Objeto.

              As Funções da Programação Modular equivalem aos Métodos da Programação Orientada a Objeto.

        \item Propriedades da Modularização

              Para manter módulos com qualidade precisamos de:

              \begin{itemize}
                  \item Encapsulamento
                  \item Coesão
                  \item Acoplamento
              \end{itemize}

        \item Encapsulamento:

              Propriedade relacionada com o nível de proteção dos elementos que compõem um módulo. Se eu tenho acesso ao ponteiro interno de uma estrutura, posso mexer com os dados armazenados em seus campos.

              Vantagens do encapsulamento:

              \begin{enumerate}

                  \item Mantém a integridade da estrutura a ser protegida.
                  \item Facilita a manutenção pois tudo que está relacionado à parte protegida encontra-se no mesmo local.

              \end{enumerate}

              Tipos de encapsulamento: Documentação, Variáveis e Código.

              \begin{itemize}

                  \item Documentação

                        \begin{itemize}

                            \item Documentação interna:

                                  Voltada para quem está programando o módulo servidor. Explica variáveis do .c e funções internas do .c. Possui pseudo instruções, que são comentários presentes no código que o explicam.

                            \item Documentação externa:

                                  Voltada para quem está progrmaando o módulo cliente. Explica as funções (objetivo, variáveis, etc). Não explica a implementação.

                            \item Documentação de uso:

                                  Voltada para o usuário. LEIAME ou README.

                        \end{itemize}

                  \item Variáveis

                        \begin{center}
                            \begin{tabular}{  c c c }
                                \hline
                                Nome      & Encapsulamento               & Tipo de Programação            \\
                                \hline
                                Global    & Não encapsulada              & Programação estruturada        \\
                                Public    & Não encapsulada              & Programação orientada a objeto \\
                                Protected & Enc. na estrutura de herança & Programação orientada a objeto \\
                                Private   & Encapsulada no objeto        & Programação orientada a objeto \\
                                Local     & Encapsulada no bloco         & Programação estruturada        \\
                                Static    & Encapsulada na classe        & Programação orientada a objeto \\
                                \hline
                            \end{tabular}
                        \end{center}

                  \item Código

                        \begin{itemize}

                            \item De iteração é encapsulado na declaração da iteração.
                            \item De função é encapsulado na função.

                        \end{itemize}

              \end{itemize}

        \item Acoplamento
        
        Propriedade relacionada com a interface entre módulos.

        OBS: Conectores são itens de interface como protótipos de função, arquivos e variáveis globais.

        Critérios de qualidade de acoplamento:

        \begin{itemize}

            \item Tamanho do conector
            
            Ex: Quantidade de parâmetros de uma função. Menos é melhor. Agrupar parâmetros em structs.

            \item Quantidade de conectores. Trabalhar com o necessário para o usuário utilizar.
            \item Complexidade dos conectores. Facilitar utilização com boa documentação.
            
        \end{itemize}

        \item Coesão
        
        Propriedade relacionada com o grau de interdependência dos elementos que compõem um módulo. Módulos tratam de um único assunto.

        Níveis de coesão:

        \begin{itemize}
            
            \item Incidental
            
            Os elementos não possuem interrelação dentro do módulo.

            \item Lógica
            
            Os elementos estão lógicamente relacionados, mesmo que por um conceito genérico.

            \item Temporal
            
            Os elementos se relacionam por serem executados em um mesmo periodo de tempo ou momento.

            \item Procedural
            
            Similar ao temporal, mas os elementos são executados em série.

            \item Funcional
            
            Os elementos estão relacionados com a funcionalidade da aplicação. Ex: Gerar relatórios.

            \item Abstração de dados
            
            Um único conceito. Ex: Módulo árvore.
            
        \end{itemize}

    \end{enumerate}

\end{doublespace}


% \section*{Question 1}

% \begin{problem}
% What is the airspeed velocity of an unladen swallow?
% \end{problem}
% \begin{center}
%     \includegraphics[width=0.5\columnwidth]{swallow.jpg} % Example image
% \end{center}

% %------------------------------------------------

% \subsection*{Answer}

% While this question leaves out the crucial element of the geographic origin of the swallow, according to Jonathan Corum, an unladen European swallow maintains a cruising airspeed velocity of \textbf{11 metres per second}, or \textbf{24 miles an hour}. The velocity of the corresponding African swallows requires further research as kinematic data is severely lacking for these species.

% %----------------------------------------------------------------------------------------

% \section*{Question 2}

% \begin{problem}
% How much wood would a woodchuck chuck if a woodchuck could chuck wood?

% \medskip

% \begin{enumerate}[(\itshape a\normalfont)] % Sub-questions styled as italic letters
%     \item Suppose ``chuck" implies throwing.
%     \item Suppose ``chuck" implies vomiting.
% \end{enumerate}
% \end{problem}

% %------------------------------------------------

% \subsection*{Answer}

% \begin{enumerate}[(\itshape a\normalfont)] % Sub-questions styled as italic letters
%     \item According to the Associated Press (1988), a New York Fish and Wildlife technician named Richard Thomas calculated the volume of dirt in a typical 25--30 foot (7.6--9.1 m) long woodchuck burrow and had determined that if the woodchuck had moved an equivalent volume of wood, it could move ``about \textbf{700 pounds (320 kg)} on a good day, with the wind at his back".

%     \item A woodchuck can ingest 361.92 cm\textsuperscript{3} (22.09 cu in) of wood per day. Assuming immediate expulsion on ingestion with a 5\% retainment rate, a woodchuck could chuck \textbf{343.82 cm\textsuperscript{3}} of wood per day.
% \end{enumerate}

% %----------------------------------------------------------------------------------------

% \section*{Question 3}

% \begin{problem}
% Identify the author of Equation \ref{eq:bayes} below and briefly describe it in Latin.

% \medskip

% \begin{equation}\label{eq:bayes}
%     P(A|B) = \frac{P(B|A)P(A)}{P(B)}
% \end{equation}

% \smallskip
% \end{problem}

% %------------------------------------------------

% \subsection*{Answer}

% Lorem ipsum dolor sit amet, consectetur adipiscing elit. Praesent porttitor arcu luctus, imperdiet urna iaculis, mattis eros. Pellentesque iaculis odio vel nisl ullamcorper, nec faucibus ipsum molestie. Sed dictum nisl non aliquet porttitor. Etiam vulputate arcu dignissim, finibus sem et, viverra nisl. Aenean luctus congue massa, ut laoreet metus ornare in. Nunc fermentum nisi imperdiet lectus tincidunt vestibulum at ac elit. Nulla mattis nisl eu malesuada suscipit.

% %----------------------------------------------------------------------------------------

% \section*{Question 4 (bonus marks)}

% \begin{problem}
% The table below shows the nutritional consistencies of two sausage types. Explain their relative differences given what you know about daily adult nutritional recommendations.

% \bigskip

% \begin{center}
%     \begin{tabular}{l l l}
%         \toprule
%         \textit{Per 50g} & Pork  & Soy   \\
%         \midrule
%         Energy           & 760kJ & 538kJ \\
%         Protein          & 7.0g  & 9.3g  \\
%         Carbohydrate     & 0.0g  & 4.9g  \\
%         Fat              & 16.8g & 9.1g  \\
%         Sodium           & 0.4g  & 0.4g  \\
%         Fibre            & 0.0g  & 1.4g  \\
%         \bottomrule
%     \end{tabular}
% \end{center}

% \medskip
% \end{problem}

% %------------------------------------------------

% \subsection*{Answer}

% Lorem ipsum dolor sit amet, consectetur adipiscing elit. Praesent porttitor arcu luctus, imperdiet urna iaculis, mattis eros. Pellentesque iaculis odio vel nisl ullamcorper, nec faucibus ipsum molestie. Sed dictum nisl non aliquet porttitor. Etiam vulputate arcu dignissim, finibus sem et, viverra nisl. Aenean luctus congue massa, ut laoreet metus ornare in. Nunc fermentum nisi imperdiet lectus tincidunt vestibulum at ac elit. Nulla mattis nisl eu malesuada suscipit.

% %----------------------------------------------------------------------------------------

% \section*{Question 5 (bonus marks)}

% \begin{problem}
% \lstinputlisting[
%     caption=Luftballons Perl Script, % Caption above the listing
%     label=lst:luftballons, % Label for referencing this listing
%     language=Perl, % Use Perl functions/syntax highlighting
%     frame=single, % Frame around the code listing
%     showstringspaces=false, % Don't put marks in string spaces
%     numbers=left, % Line numbers on left
%     numberstyle=\tiny, % Line numbers styling
% ]{luftballons.pl}

% \begin{enumerate}
%     \item How many luftballons will be output by the Listing \ref{lst:luftballons} above?
%     \item Identify the regular expression in Listing \ref{lst:luftballons} and explain how it relates to the anti-war sentiments found in the rest of the script.
% \end{enumerate}

% \end{problem}

% %------------------------------------------------

% \subsection*{Answer}

% \begin{enumerate}
%     \item 99 luftballons.
%     \item Lorem ipsum dolor sit amet, consectetur adipiscing elit. Praesent porttitor arcu luctus, imperdiet urna iaculis, mattis eros. Pellentesque iaculis odio vel nisl ullamcorper, nec faucibus ipsum molestie. Sed dictum nisl non aliquet porttitor. Etiam vulputate arcu dignissim, finibus sem et, viverra nisl. Aenean luctus congue massa, ut laoreet metus ornare in. Nunc fermentum nisi imperdiet lectus tincidunt vestibulum at ac elit. Nulla mattis nisl eu malesuada suscipit.
% \end{enumerate}

% %----------------------------------------------------------------------------------------

\end{document}
