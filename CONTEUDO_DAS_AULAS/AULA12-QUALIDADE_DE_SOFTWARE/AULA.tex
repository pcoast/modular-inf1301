%%%%%%%%%%%%%%%%%%%%%%%%%%%%%%%%%%%%%%%%%
% fphw Assignment
% LaTeX Template
% Version 1.0 (27/04/2019)
%
% This template originates from:
% https://www.LaTeXTemplates.com
%
% Authors:
% Class by Felipe Portales-Oliva (f.portales.oliva@gmail.com) with template 
% content and modifications by Vel (vel@LaTeXTemplates.com)
%
% Template (this file) License:
% CC BY-NC-SA 3.0 (http://creativecommons.org/licenses/by-nc-sa/3.0/)
%
%%%%%%%%%%%%%%%%%%%%%%%%%%%%%%%%%%%%%%%%%

%----------------------------------------------------------------------------------------
%	PACKAGES AND OTHER DOCUMENT CONFIGURATIONS
%----------------------------------------------------------------------------------------

\documentclass[
	12pt, % Default font size, values between 10pt-12pt are allowed
	%letterpaper, % Uncomment for US letter paper size
	%spanish, % Uncomment for Spanish
]{fphw}

% Template-specific packages
\usepackage[utf8]{inputenc} % Required for inputting international characters
\usepackage[T1]{fontenc} % Output font encoding for international characters
\usepackage{mathpazo} % Use the Palatino font
\usepackage{setspace}

\usepackage{graphicx} % Required for including images

\usepackage{booktabs} % Required for better horizontal rules in tables

\usepackage{listings} % Required for insertion of code

\usepackage{enumerate} % To modify the enumerate environment

\usepackage{parskip}% http://ctan.org/pkg/parskip

\usepackage{amsmath}

\usepackage{amssymb}

\usepackage{caption}

\usepackage{tikz}
\newcommand*\circled[1]{\tikz[baseline=(char.base)]{
            \node[shape=circle,draw,inner sep=2pt] (char) {#1};}}

            \usetikzlibrary{decorations.pathreplacing,positioning, arrows.meta}

            \newcommand{\ImageWidth}{\textwidth}

\usepackage{enumitem}

\usepackage[ruled,portuguese,onelanguage,longend]{algorithm2e} %for psuedo code}% http://ctan.org/pkg/algorithm2e
% \makeatletter
% \renewcommand{\@algocf@capt@plain}{above}% formerly {bottom}
% \makeatother

\usepackage{multicol}
%----------------------------------------------------------------------------------------
%	ASSIGNMENT INFORMATION
%----------------------------------------------------------------------------------------

\title{AULA 12 - Qualidade de Software} % Assignment title

\author{AVC, JPP, PC} % Student name

\date{} % Due date

\institute{Pontifícia Universidade Católica do Rio de Janeiro \\ Departamento de Informática} % Institute or school name

\class{Programação Modular (INF1301)} % Course or class name

\professor{Flavio Bevilacqua} % Professor or teacher in charge of the assignment

%----------------------------------------------------------------------------------------

\begin{document}

\maketitle % Output the assignment title, created automatically using the information in the custom commands above

%----------------------------------------------------------------------------------------
%	ASSIGNMENT CONTENT
%----------------------------------------------------------------------------------------
\begin{doublespace}

    \begin{enumerate}[label=\textbf{\arabic*)}]

        \item \textbf{Definição:}

              Qualidade é aproximar o produto gerado à necessidade do cliente.

              Boa qualidade indica que o objeto possui características que o cliente considera positivas. Qualidade satisfatória indica que o objeto possui as características exatas que o cliente deseja.

        \item \textbf{Controle de Qualidade}

              O controle de qualidade não garante qualidade satisfatória pois não consegue definir o que a aplicação deveria fazer.

        \item \textbf{Tipos de Qualidade}

              \begin{itemize}

                  \item Requerida: aquilo que o cliente quer. Parâmetro da qualidade efetiva.
                  \item Efetiva: o que o artefato efetivamente possui. Parâmetro da qualidade atestada. Queremos aproximar a qualidade efetiva da qualidade requerida.
                  \item Atestada: qualidade atestada por um controle de qualidade. Parâmetro da qualidade observada. Queremos aproximar a qualidade atestada da qualidade efetiva.
                  \item Observada: análise feita a partir do resultado do controle. Queremos aproximar a qualidade observada da qualidade atestada.

              \end{itemize}

              Quando o programa possui erros que não foram testados, então a qualidade efetiva será menor do que a qualidade atestada.

              Quando a análise do resultado do controle de qualidade é incorreta, a qualidade observada será diferente da atestada.

              Quando não testamos tudo que o cliente requeriu, a qualidade atestada é maior do que a qualidade requerida.


    \end{enumerate}


\end{doublespace}
%----------------------------------------------------------------------------------------

\end{document}
