%%%%%%%%%%%%%%%%%%%%%%%%%%%%%%%%%%%%%%%%%
% fphw Assignment
% LaTeX Template
% Version 1.0 (27/04/2019)
%
% This template originates from:
% https://www.LaTeXTemplates.com
%
% Authors:
% Class by Felipe Portales-Oliva (f.portales.oliva@gmail.com) with template 
% content and modifications by Vel (vel@LaTeXTemplates.com)
%
% Template (this file) License:
% CC BY-NC-SA 3.0 (http://creativecommons.org/licenses/by-nc-sa/3.0/)
%
%%%%%%%%%%%%%%%%%%%%%%%%%%%%%%%%%%%%%%%%%

%----------------------------------------------------------------------------------------
%	PACKAGES AND OTHER DOCUMENT CONFIGURATIONS
%----------------------------------------------------------------------------------------

\documentclass[
	12pt, % Default font size, values between 10pt-12pt are allowed
	%letterpaper, % Uncomment for US letter paper size
	%spanish, % Uncomment for Spanish
]{fphw}

% Template-specific packages
\usepackage[utf8]{inputenc} % Required for inputting international characters
\usepackage[T1]{fontenc} % Output font encoding for international characters
\usepackage{mathpazo} % Use the Palatino font
\usepackage{setspace}

\usepackage{graphicx} % Required for including images

\usepackage{booktabs} % Required for better horizontal rules in tables

\usepackage{listings} % Required for insertion of code

\usepackage{enumerate} % To modify the enumerate environment
\usepackage{parskip}

%----------------------------------------------------------------------------------------
%	ASSIGNMENT INFORMATION
%----------------------------------------------------------------------------------------

\title{AULA 4 - Processo de desenvolvimento} % Assignment title

\author{AVC, JPP, PC} % Student name

\date{} % Due date

\institute{Pontifícia Universidade Católica do Rio de Janeiro \\ Departamento de Informática} % Institute or school name

\class{Programação Modular (INF1301)} % Course or class name

\professor{Flavio Bevilacqua} % Professor or teacher in charge of the assignment

%----------------------------------------------------------------------------------------

\begin{document}

\maketitle % Output the assignment title, created automatically using the information in the custom commands above

%----------------------------------------------------------------------------------------
%	ASSIGNMENT CONTENT
%----------------------------------------------------------------------------------------
\begin{doublespace}
O foco da carreira tecnolóica é o cliente, sempre.

Requisitos -> Análise e projeto -> Implementação -> Testes -> Homologação -> Implantação

O Analista de Negócios faz intermédio entre o cliente e a equipe de desenvolvimento.

O Lider de Projeto mantém coesão na equipe de desenvolvimento, ele coordena. O Lider de Projeto tambêm arquiteta o planejamento, levando em conta o esforço e o prazo. Cria-se um cronograma.

Prazo polîtico é inegociável. Caso o cliente dê um prazo político, o Lider de Projeto negocia hora extra. 

O Lider de Projeto faz um acompanhamento do projeto, verificando se as etapas estão completas em seus devidos prazos.

Se não for possível entregar o projeto todo no prazo e o prazo não puder ser estendido, fazemos uma entrega parcial.

O Lider de Projeto deve manejar riscos. Por exemplo: gravidez, greve do cliente, equipe despreparada, pessoas que moram longe ou podem se acidentar.

O Analista de Requisitos deve elicitar (buscar informações com o cliente sobre o que ele quer), documentar a elicitação, verificar (ver com a equipe de desenvolvimento se o que o cliente deseja é factível) e validar (checar com o cliente).

O Analista de sistemas produz modelos, telas e lógicas dos programas. Ele também faz o projeto lógico (relações e bancos de dados) e o projeto físico.

O Implementador conhece a linguagem muito bem. Ele programa e faz o teste unitário.

Na fase de testes, faz-se o teste integrado. O melhor testador é aquele que tem algo contra você. Dependendo do teste, pode-se voltar para a etapa de requistos, quando o programa não faz o que o requisito pede, de análise e projeto, quando há um modelo equivocado, ou implementação, quando o modelo não foi traduzido corretamente no código.

Na Homologação, o cliente testa uma build do projeto. Erros encontrados nesta etapa podem voltar para qualquer uma das etapas anteriores. Um erro relatado pelo cliente volta para a etapa de requisitos quando o que ele pediu não condiz com a documentação de requisitos. Quando a sugestão do cliente envolve algo a mais que ele deseja, adicional à documentação de requisitos, pode-se optar por um retrabalho, implementando a sugestão no projeto atual, ou outro projeto, deixando para depois a implementação da sugestão e finalizando o projeto atual.

Por fim, na fase de implantação o sitema é posto para funcionar em ambiente de produção.

O Gestor de Qualidade garante que a empresa trabalhe com qualidade diariamente e produza aplicações com qualidade. Ele corrige o trabalho de todos e acompanha todas as etapas, garantindo que tudo está nos conformes e padrões.

O Gerente de Configurações maneja baselines e ambientes de desenvolvimento.

\end{doublespace}
%----------------------------------------------------------------------------------------

\end{document}
