%%%%%%%%%%%%%%%%%%%%%%%%%%%%%%%%%%%%%%%%%
% fphw Assignment
% LaTeX Template
% Version 1.0 (27/04/2019)
%
% This template originates from:
% https://www.LaTeXTemplates.com
%
% Authors:
% Class by Felipe Portales-Oliva (f.portales.oliva@gmail.com) with template 
% content and modifications by Vel (vel@LaTeXTemplates.com)
%
% Template (this file) License:
% CC BY-NC-SA 3.0 (http://creativecommons.org/licenses/by-nc-sa/3.0/)
%
%%%%%%%%%%%%%%%%%%%%%%%%%%%%%%%%%%%%%%%%%

%----------------------------------------------------------------------------------------
%	PACKAGES AND OTHER DOCUMENT CONFIGURATIONS
%----------------------------------------------------------------------------------------

\documentclass[
	12pt, % Default font size, values between 10pt-12pt are allowed
	%letterpaper, % Uncomment for US letter paper size
	%spanish, % Uncomment for Spanish
]{fphw}

% Template-specific packages
\usepackage[utf8]{inputenc} % Required for inputting international characters
\usepackage[T1]{fontenc} % Output font encoding for international characters
\usepackage{mathpazo} % Use the Palatino font
\usepackage{setspace}

\usepackage{graphicx} % Required for including images

\usepackage{booktabs} % Required for better horizontal rules in tables

\usepackage{listings} % Required for insertion of code

\usepackage{enumerate} % To modify the enumerate environment

%----------------------------------------------------------------------------------------
%	ASSIGNMENT INFORMATION
%----------------------------------------------------------------------------------------

\title{AULA - Especificação de requisitos} % Assignment title

\author{AVC, JPP, PC} % Student name

\date{} % Due date

\institute{Pontifícia Universidade Católica do Rio de Janeiro \\ Departamento de Informática} % Institute or school name

\class{Programação Modular (INF1301)} % Course or class name

\professor{Flavio Bevilacqua} % Professor or teacher in charge of the assignment

%----------------------------------------------------------------------------------------

\begin{document}

\maketitle % Output the assignment title, created automatically using the information in the custom commands above

%----------------------------------------------------------------------------------------
%	ASSIGNMENT CONTENT
%----------------------------------------------------------------------------------------
\begin{doublespace}

As demandas do cliente são transformadas em documento.

\begin{enumerate}

    \item Requisito
    
    \begin{itemize}

        \item O que deve ser feito.
        \item Não descreve como fazer.
        \item Itens curtos e diretos.
        \item Em linguagem natural. O problema é a ambiguidade.
        \item Faz parte do contrato assinado junto ao cliente. Caso o cliente decida que o que foi feito não é o que ele pediu, temos refazer de graça.
        
    \end{itemize}

    \item Etapas
    
    \begin{itemize}

        \item Elicitação
        
        Conversa inicial onde o cliente relata o que ele quer. Técnicas: reunião (entrevista), brainstorming, questionário. O resultado da elicitação é uma ata. Na elicitação, se o assunto entrar numa tangente desnecessária para o desenvolvimento do sistema, devemos retomar ao elementar rapidamente.
        
        OBS: Escopo de efeito - requisito mais abrangente vs requisito mais específico. Na ata estarão escritos requisitos abrangentes, como "quero uma aplicação de labirinto", e os mais específicos, como "quero que use o algorimo X".

        \item Documentação
        
        Gerar um documento organizando o conteúdo adquirido na etapa de elicitação. Podemos durante esta etapa, voltar para a etapa de elicitação. É importante retirar todas as ambiguidades e partes pouco claras do documento.

        \item Verificação
        
        Participam os técnicos (equipe que desenvolve) para determinar se o que foi pedido e documentado é plausível e possível de desenvolver.

        \item Validação pelo cliente
        
        Clinente confirma a documentação escrita.

    \end{itemize}

    \item Tipos de requisito
    
    \begin{itemize}

        \item Requisitos funcionais
        
        O que deve ser feito de acordo com o objetvio da aplicação (funcionalidades e regras de negócio).

        \item Requisitos não funcionais
        
        Requisitos que não têm relação com um a funcionalidade específica. Conjunto de propriedades que a aplicação deve possuir. Por exemplo, disponibilidade 24x7, segurança com login, disponibilidade de tempo de resposta.

        \item Requisito inverso
        
        Aquilo que não será feito. Especificado para combater ambiguidades e partes pouco claras.

        \item Hipóteses
        
        Regras consideradas válidas antes do desenvolvimento da aplicação. Premisas que definem aquilo que precisa ou não ser feito.

        \item Restrições
        
        Regras que restringem as alternativas de resolução de um problema.
        
    \end{itemize}

    \item Exemplos de Requisitos
    
    \begin{itemize}

        \item Bem formulados:

        - "os dados armazenados de aluno são matrícula e nome."

        - "tempo de resposta das consultas não pode ultrapassar 2 segundos."

        \item Mal formulados:
        
        - "a interface deve ser de fácil utilização pelo usuário."
        
        - "a consulta tem que ser rápida."

    \end{itemize}

\end{enumerate}

\end{doublespace}
    %----------------------------------------------------------------------------------------

\end{document}
