\chapter{Requerimentos de Interface Externa}

\section{Interface de Usuârio}
% $<$Describe the logical characteristics of each interface between the software
% product and the users. This may include sample screen images, any GUI standards
% or product family style guides that are to be followed, screen layout
% constraints, standard buttons and functions (e.g., help) that will appear on
% every screen, keyboard shortcuts, error message display standards, and so on.
% Define the software components for which a user interface is needed. Details of
% the user interface design should be documented in a separate user interface
% specification.$>$

O usuário deve ser capaz de construir o labirinto manulamente, escrevendo em um arquivo de texto que será salvo e, posteriormente, lido pela aplicação. O labirinto e sua solução devem ser gerados pela aplicação e mostrados para o usuário graficamente utilizando caracteres ascii para compor a figura. Haverá um menu possibilitando que o usuário possa escolher modificar o labirinto ou resolver quantas vezes desejar.

% \section{Hardware Interfaces}
% $<$Describe the logical and physical characteristics of each interface between
% the software product and the hardware components of the system. This may include
% the supported device types, the nature of the data and control interactions
% between the software and the hardware, and communication protocols to be
% used.$>$

% \section{Software Interfaces}
% $<$Describe the connections between this product and other specific software
% components (name and version), including databases, operating systems, tools,
% libraries, and integrated commercial components. Identify the data items or
% messages coming into the system and going out and describe the purpose of each.
% Describe the services needed and the nature of communications. Refer to
% documents that describe detailed application programming interface protocols.
% Identify data that will be shared across software components. If the data
% sharing mechanism must be implemented in a specific way (for example, use of a
% global data area in a multitasking operating system), specify this as an
% implementation constraint.$>$

% \section{Communications Interfaces}
% $<$Describe the requirements associated with any communications functions
% required by this product, including e-mail, web browser, network server
% communications protocols, electronic forms, and so on. Define any pertinent
% message formatting. Identify any communication standards that will be used, such
% as FTP or HTTP. Specify any communication security or encryption issues, data
% transfer rates, and synchronization mechanisms.$>$
