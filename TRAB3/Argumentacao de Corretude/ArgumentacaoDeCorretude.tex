\documentclass[a4paper,12pt,oneside]{book}
%
\let\proof\relax
\let\endproof\relax
\usepackage{amsthm}
\usepackage{graphicx,rotating}
\usepackage{complexity}
\usepackage{tikz}
\usepackage{xspace}
\usepackage{fix-cm}
\usetikzlibrary{shadows}
\usetikzlibrary{shapes,positioning,arrows}
\usetikzlibrary{decorations.pathmorphing}
\usetikzlibrary{shadows,arrows,patterns,automata,calc,intersections,fit,shapes.misc, decorations.markings}
\usepackage[underline=false,rounded corners=true]{pgf-umlsd}
\usepackage{todonotes}
\usepackage{paralist}
%\usepackage{parskip} % Creates paragraphs - Maybe remove for more space
\setdefaultenum{1)}{(a)}{i)}{A)}
\usepackage{multicol}% http://ctan.org/pkg/multicols
\usepackage{multirow}

\usepackage{lmodern}
\usepackage[brazilian]{babel}

% \usepackage{msc}
% \usepackage[
% n,
% operators,
% advantage,
% sets,
% adversary,
% landau,
% probability,
% notions,
% logic,
% ff,
% mm,
% primitives,
% events,
% complexity,
% asymptotics,
% keys]{cryptocode}
% \usepackage{dashbox}

\usepackage[
%pdfauthor={},
%pdfsubject={},
%pdftitle={},
%pdfkeywords={},
bookmarks=false,
breaklinks=true,
colorlinks=true,
linkcolor=black,
citecolor=black,
urlcolor=black,
%pdfstartpage=19,
pdfpagelayout=SinglePage
]{hyperref}
%enables correct jumping to figures when referencing
\usepackage[all]{hypcap}

\usepackage{dsfont}
\usepackage[bottom]{footmisc}
\usepackage[nocompress]{cite}
\usepackage{listings}
\usepackage{fancyvrb}
\usepackage{bm}
\usepackage{amssymb}
\usepackage{amsmath}
\usepackage{cleveref}
\usepackage{mathtools}
% \usepackage{algorithm}
% \usepackage{algorithmic}
\usepackage{etoolbox}
\usepackage{varwidth} %for the varwidth minipage environment
\usepackage{array} % for defining a new column type
\usepackage{chngcntr}
\usepackage{stmaryrd}
\usepackage{tabulary}
\usepackage{booktabs}

\usepackage{wrapfig,epsfig}

\usepackage{xcolor}% http://ctan.org/pkg/xcolor
\usepackage{graphicx}
\usepackage{color}
\usepackage{todonotes}
% Fonts
\usepackage{microtype}
\usepackage[T1]{fontenc}
\usepackage{textcomp}
\usepackage[utf8]{inputenc}
\usepackage{inconsolata}

\usepackage{enumitem}

%B-Method packages
% \usepackage{eventB}
% \usepackage{zed-csp}

% \usepackage[linesnumbered, ruled, noend]{algorithm2e} % algorithms
% \usepackage{algorithmicx}

\usepackage{float}
\usepackage{listing}
\usepackage{mdframed}
% \usepackage{subcaption}
\usepackage{caption}
% \usepackage{draftwatermark}
% \SetWatermarkScale{1}

% Referencing
\newcommand{\Fref}[1]{Figure~\ref{#1}}
\newcommand{\fref}[1]{figure~\ref{#1}}

\newcommand{\Tref}[1]{Table~\ref{#1}}
\newcommand{\tref}[1]{table~\ref{#1}}

\newcommand{\Eref}[1]{Equation~\ref{#1}}
\newcommand{\eref}[1]{equation~\ref{#1}}

\newcommand{\Sref}[1]{Section~\ref{#1}}
\newcommand{\sref}[1]{section~\ref{#1}}

\newcommand{\Lref}[1]{Listing~\ref{#1}}
\newcommand{\lref}[1]{listing~\ref{#1}}

% names
\newcommand{\otoken}{O\textsc{-token}\xspace}
\newcommand{\otokens}{O\textsc{-tokens}\xspace}
\newcommand{\otokenFull}{O\textsc{bscure} T\textsc{oken}\xspace}

% Messages
\newcommand{\osrreq}{O\textsc{sr-req}\xspace}
\newcommand{\osrconf}{O\textsc{sr-conf}\xspace}


% SRS cover page
\title{
  \begin{flushright}
  \Huge{ARGUMENTAÇÃO \\ DE CORRETUDE}\\
  da aplicação\\
  LABIRINTO\\
  ~\\
  \LARGE{Versão 1.0}\\
  ~\\
  INF1301 - Programação Modular\\ DI/PUC-Rio
  \end{flushright}
}

% Author
\author{Antônio Chaves - AVC\\João Pedro Paiva - JPP\\Pedro Costa - PC}
\date{\today}

\begin{document}

\setlength{\parindent}{0in}

\setlength{\parskip}{\baselineskip}


\frontmatter
\maketitle

\tableofcontents

\chapter*{Histórico de Revisões}

\begin{center}
    \begin{tabular}{|c|c|c|c|}
        \hline
        Versão & Data       & Autor & Observações                                      \\
        \hline
        1.0    & 08/10/2019 & AVC   & Versão do Trab2                                  \\
        \hline
    \end{tabular}
\end{center}


\mainmatter

\chapter{Pseudocódigo}

\begin{algorithm}[h]

    \caption{MAT\_tpCondRet MAT\_vaiParaPos(MAT\_tppMatriz CabecaDaMatriz, char Coluna, char Linha)}

    \SetAlgoLined
    % \KwData{this text}
    % \KwResult{how to write algorithm with \LaTeX2e }
    AE $\longrightarrow$

    \Indp\Inicio
    {

        NÓ-CORRENTE $\longleftarrow$ NÓ-PRIMEIRO

        \Enqto{COLUNA > 0}
        {

            COLUNA $\longleftarrow$ COLUNA - 1

            MAT\_vaiParaDireita(CabecaDaMatriz)
        }

        \Enqto{LINHA > 0}
        {

            LINHA $\longleftarrow$ LINHA - 1

            MAT\_vaiParaBaixo(CabecaDaMatriz)
        }

        \Retorna{MAT\_CondRetOK}

    }

    \Indm AS $\longrightarrow$

\end{algorithm}

\chapter{Argumentação de Sequência 1}

AE: A posição desejada pertence à matriz. Nó corrente da matriz não aponta necessariamente para a posição inicial (mesma posição que o primeiro nó). Cabeça da matriz != NULL. Valem as assertivas estruturais da matriz com cabeça.

AS: Nó corrente da matriz está na posição desejada. Valem as assertivas estruturais da matriz com cabeça.

AI 1: Nó corrente da matriz aponta para o primeiro nó da matriz.

AI 2: Nó corrente aponta para a posição com a coluna desejada.

AI 3: Nó corrente aponta para a posição com a linha desejada.

\chapter{Argumentação de Repetição 1}

AE: AI 1.

AS: AI 2.

AINV:

\begin{itemize}

    \item Existem dois conjuntos: colunas percorridas e colunas que restam percorrer.
    \item COLUNA indica o número de colunas que restam percorrer até chegar à coluna desejada.

\end{itemize}

\begin{enumerate}[label=\protect\circled{\arabic*}]
    \item AE $\Longrightarrow$ AINV

          \begin{itemize}
              \item Pela AE, o nó corrente da matriz aponta para o primeiro nó da matriz. Todos os elementos estão no conjunto colunas que restam percorrer e o conjunto colunas percorridas está vazio. Logo, vale a AINV.
          \end{itemize}

    \item AE \&\& (Condição == False) $\Longrightarrow$ AS

          \begin{itemize}
              \item Pela AE, o nó corrente da matriz aponta para o primeiro nó da matriz. Para que (Condição == False), COLUNA == 0. Logo, o conjunto colunas que restam percorrer é vazio, ou seja, o nó corrente aponta para a posição com a coluna desejada. Portanto, vale a AS.
          \end{itemize}

    \item AE \&\& (Condição == True) \circled{+} B $\Longrightarrow$ AINV

          \begin{itemize}
              \item Pela AE, o nó corrente da matriz aponta para o primeiro nó da matriz. Para que  (Condição == True), COLUNA > 0. O nó corrente passará a apontar para a posição da coluna seguinte, ou seja, uma coluna passará do conjunto colunas que restam percorrer para o conjunto colunas percorridas. COLUNA será decrementada. Com isso, os dois conjuntos existem e COLUNA indica o número de colunas que restam percorrer até chegar à coluna desejada. Logo, vale a AINV.
          \end{itemize}

    \item AINV \&\& (Condição == True) \circled{+} B $\Longrightarrow$ AINV

          \begin{itemize}
              \item Para que a AINV continue valendo, B deve garantir que um dos elementos do conjuntocolunas que restam percorrer passe para o conjunto colunas percorridas, e COLUNA seja decrementada.
          \end{itemize}

    \item AINV \&\& (Condição == False) \circled{+} B $\Longrightarrow$ AS

          \begin{itemize}
              \item Se (Condição == False), então COLUNA == 0. Logo, o conjunto colunas que restam percorrer é vazio, ou seja, o nó corrente aponta para a posição com a coluna desejada, vale a AS.
          \end{itemize}

    \item Término

          \begin{itemize}
              \item Como a cada ciclo, um dos elementos do conjunto colunas que restam percorrer é retirado, e este conjunto possui um número finito de elementos, a repetição terminará em um número finito de passos.
          \end{itemize}

\end{enumerate}

\chapter{Argumentação de Repetição 2}

AE: AI 2.

AS: AI 3.

AINV:

\begin{itemize}

    \item Existem dois conjuntos: linhas percorridas e linhas que restam percorrer.
    \item LINHA indica o número de linhas que restam percorrer até chegar à linha desejada.

\end{itemize}

\begin{enumerate}[label=\protect\circled{\arabic*}]
    \item AE $\Longrightarrow$ AINV

          \begin{itemize}
              \item Pela AE, o nó corrente da matriz aponta para o primeiro nó da matriz. Todos os elementos estão no conjunto linhas que restam percorrer e o conjunto linhas percorridas está vazio. Logo, vale a AINV.
          \end{itemize}

    \item AE \&\& (Condição == False) $\Longrightarrow$ AS

          \begin{itemize}
              \item Pela AE, o nó corrente da matriz aponta para o primeiro nó da matriz. Para que (Condição == False), LINHA == 0. Logo, o conjunto linhas que restam percorrer é vazio, ou seja, o nó corrente aponta para a posição com a linha desejada. Portanto, vale a AS.
          \end{itemize}

    \item AE \&\& (Condição == True) \circled{+} B $\Longrightarrow$ AINV

          \begin{itemize}
              \item Pela AE, o nó corrente da matriz aponta para o primeiro nó da matriz. Para que  (Condição == True), LINHA > 0. O nó corrente passará a apontar para a posição da linha seguinte, ou seja, uma linha passará do conjunto linhas que restam percorrer para o conjunto linhas percorridas. LINHA será decrementada. Com isso, os dois conjuntos existem e LINHA indica o número de linhas que restam percorrer até chegar à linha desejada. Logo, vale a AINV.
          \end{itemize}

    \item AINV \&\& (Condição == True) \circled{+} B $\Longrightarrow$ AINV

          \begin{itemize}
              \item Para que a AINV continue valendo, B deve garantir que um dos elementos do conjuntolinhas que restam percorrer passe para o conjunto linhas percorridas, e LINHA seja decrementada.
          \end{itemize}

    \item AINV \&\& (Condição == False) \circled{+} B $\Longrightarrow$ AS

          \begin{itemize}
              \item Se (Condição == False), então LINHA == 0. Logo, o conjunto linhas que restam percorrer é vazio, ou seja, o nó corrente aponta para a posição com a linha desejada, vale a AS.
          \end{itemize}

    \item Término

          \begin{itemize}
              \item Como a cada ciclo, um dos elementos do conjunto linhas que restam percorrer é retirado, e este conjunto possui um número finito de elementos, a repetição terminará em um número finito de passos.
          \end{itemize}

\end{enumerate}

\chapter{Pseudocódigo}

\begin{algorithm}[h]

    \caption{MAT\_tpCondRet MAT\_obterElemento(MAT\_tppMatriz CabecaDaMatriz, void **elemento)}

    \SetAlgoLined
    % \KwData{this text}
    % \KwResult{how to write algorithm with \LaTeX2e }
    AE $\longrightarrow$

    \Indp\Inicio
    {

        \lSe{$\nexists$ELEMENTO-DO-NÓ-CORRENTE}
        {

            \Retorna MAT\_CondRetNoVazio
        }

        ELEMENTO $\longleftarrow$ ELEMENTO-DO-NÓ-CORRENTE

        \Retorna{MAT\_CondRetOK}

    }

    \Indm AS $\longrightarrow$

\end{algorithm}

\chapter{Argumentação de Seleção 1}

AE: Ponteiro corrente aponta para o nó de onde deseja-se obter o conteúdo. Conteúdo do nó corrente pode existir ou não. Cabeça da matriz != NULL. Valem as assertivas estruturais da matriz com cabeça.


AS: Elemento foi obtido do nó corrente da matriz ou elemento não existe (é nulo). Valem as assertivas estruturais da matriz com cabeça.

\begin{enumerate}[label=\protect\circled{\arabic*}]
    \item AE \&\& (Condição == True) \circled{+} B1 $\Longrightarrow$ AS

          Pela AE, o conteúdo do nó corrente pode existir ou não.. Como (Condição == True), não existe elemento no nó corente (conteúdo é nulo). Retornamos a condição de nó vazio. Já que o elemento é nulo, vale a AS.

    \item AE \&\& (Condição == False) \circled{+} B2 $\Longrightarrow$ AS

    Pela AE, conteúdo do nó corrente pode existir ou não. Como (Condição == False), existe elemento no nó corente (conteúdo não é nulo). Obtemos o elemento do nó corrente, logo, vale a AS.

\end{enumerate}

\chapter{Argumentação de Sequência 1}

AE: O ponteiro corrente aponta para o nó de onde deseja-se obter o conteúdo e existe elemento no nó corente (conteúdo não é nulo).


AS: AS geral.


AI 1: ELEMENTO aponta para o conteúdo do nó corrente.

\end{document}
