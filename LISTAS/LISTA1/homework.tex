\documentclass[]{article}
\usepackage[brazilian]{babel}
\usepackage[utf8]{inputenc}
\usepackage[T1]{fontenc}
\usepackage{fancyhdr}
\usepackage{extramarks}
\usepackage{amsmath}
\usepackage{amsthm}
\usepackage{amsfonts}
\usepackage{tikz}
\usepackage[plain]{algorithm}
\usepackage{algpseudocode}

\usetikzlibrary{automata,positioning}

%
% Basic Document Settings
%

\topmargin=-0.45in
\evensidemargin=0in
\oddsidemargin=0in
\textwidth=6.5in
\textheight=9.0in
\headsep=0.25in

\linespread{1.1}

\pagestyle{fancy}
\lhead{AVC, JPP, PC}
\chead{\hmwkClass\ (\hmwkClassInstructor\ \hmwkClassTime): \hmwkTitle}
\rhead{\firstxmark}
\lfoot{\lastxmark}
\cfoot{\thepage}

\renewcommand\headrulewidth{0.4pt}
\renewcommand\footrulewidth{0.4pt}

\setlength\parindent{0pt}

%
% Create Questão Sections
%

\newcommand{\enterProblemHeader}[1]{
    \nobreak\extramarks{}{Questão \arabic{#1} continued on next page\ldots}\nobreak{}
    \nobreak\extramarks{Questão \arabic{#1} (continued)}{Questão \arabic{#1} continued on next page\ldots}\nobreak{}
}

\newcommand{\exitProblemHeader}[1]{
    \nobreak\extramarks{Questão \arabic{#1} (continued)}{Questão \arabic{#1} continued on next page\ldots}\nobreak{}
    \stepcounter{#1}
    \nobreak\extramarks{Questão \arabic{#1}}{}\nobreak{}
}

\setcounter{secnumdepth}{0}
\newcounter{partCounter}
\newcounter{homeworkProblemCounter}
\setcounter{homeworkProblemCounter}{1}
\nobreak\extramarks{Questão \arabic{homeworkProblemCounter}}{}\nobreak{}

%
% Homework Questão Environment
%
% This environment takes an optional argument. When given, it will adjust the
% problem counter. This is useful for when the problems given for your
% assignment aren't sequential. See the last 3 problems of this template for an
% example.
%
\newenvironment{homeworkProblem}[1][-1]{
    \ifnum#1>0
        \setcounter{homeworkProblemCounter}{#1}
    \fi
    \section{Questão \arabic{homeworkProblemCounter}}
    \setcounter{partCounter}{1}
    \enterProblemHeader{homeworkProblemCounter}
}{
    \exitProblemHeader{homeworkProblemCounter}
}

%
% Homework Details
%   - Title
%   - Due date
%   - Class
%   - Section/Time
%   - Instructor
%   - Author
%

\newcommand{\hmwkTitle}{Lista 1}
\newcommand{\hmwkDueDate}{February 12, 2014}
\newcommand{\hmwkClass}{Programação Modular}
\newcommand{\hmwkClassTime}{}
\newcommand{\hmwkClassInstructor}{Professor Flavio Bevilacqua}


\newcommand{\hmwkAuthorName}{
\begin{tabular}{c} Antônio Vasconcellos Chaves \\ Engenharia da Computaçāo \\ Pontifícia Universidade Católica \\ do Rio de Janeiro \\ Rio de Janeiro, RJ 22451-900 \\ antoniovasconcelloschaves@gmail.com \end{tabular} \and
\begin{tabular}{c} João Pedro Paiva \\ Ciência da Computaçāo \\ Pontifícia Universidade Católica \\ do Rio de Janeiro \\ Rio de Janeiro, RJ 22451-900 \\ joaopedrordepaiva@gmail.com \end{tabular} \and
\\\begin{tabular}{c} Pedro Moreira Costa \\ Engenharia da Computaçāo \\ Pontifícia Universidade Católica \\ do Rio de Janeiro \\ Rio de Janeiro, RJ 22451-900 \\ pedromoreiramcosta@gmail.com \end{tabular}
}

%
% Title Page
%

\title{
    \vspace{2in}
    \textmd{\textbf{\hmwkClass:\ \hmwkTitle}}\\
    \normalsize\vspace{0.1in}\small{Entrega\ no dia\ \hmwkDueDate\ às 17h}\\
    \vspace{0.1in}\large{\textit{\hmwkClassInstructor\ \hmwkClassTime}}
    \vspace{1.5in}
}

\author{\hmwkAuthorName}
\date{}

\renewcommand{\part}[1]{\textbf{\large Parte \Alph{partCounter}}\stepcounter{partCounter}\\}

%
% Various Helper Commands
%

% Useful for algorithms
\newcommand{\alg}[1]{\textsc{\bfseries \footnotesize #1}}

% For derivatives
\newcommand{\deriv}[1]{\frac{\mathrm{d}}{\mathrm{d}x} (#1)}

% For partial derivatives
\newcommand{\pderiv}[2]{\frac{\partial}{\partial #1} (#2)}

% Integral dx
\newcommand{\dx}{\mathrm{d}x}

% Alias for the Solution section header
\newcommand{\solution}{\textbf{\large Solução}}

% Probability commands: Expectation, Variance, Covariance, Bias
\newcommand{\E}{\mathrm{E}}
\newcommand{\Var}{\mathrm{Var}}
\newcommand{\Cov}{\mathrm{Cov}}
\newcommand{\Bias}{\mathrm{Bias}}

\begin{document}

\maketitle

\pagebreak

\begin{homeworkProblem}

    Explique com um exemplo o conceito de callback.

    \vspace{\baselineskip}
    \textbf{Solução}

	Callback é o termo usado para quando ao ter os dados requisitados, enviados pelo cliente, ainda há requisitos não preenchidos que não são obrigatórios. Por exemplo, em um cadastro, o servidor precisa das informações (nome, sobrenome e CPF) do cliente. O cliente esquece de preencher o CPF. Então o servidor volta/gera para interface com um aviso de que aquele campo é de preenchimento obrigatório.

\end{homeworkProblem}


\begin{homeworkProblem}

    Apresente um requisito funcional bem formulado, derivado de um requisito não funcional, diferente do exemplo de login visto em aula.

    \vspace{\baselineskip}
    \textbf{Solução}
	
	A necessidade que a solução seja descentralizada, por exemplo. Pois torna necessária a implementação de módulos para estabelecer a conexão entre os nós do sistema distribuído.

\end{homeworkProblem}


\begin{homeworkProblem}

    Um bom acoplamento resulta em um bom encapsulamento. Certo, errado, tipo assim, justifique.

    \vspace{\baselineskip}
    \textbf{Solução}

	Errado. Ambos estão relacionados mas não têm influência um sobre o outro. O acoplamento pode ser feito com diversos parâmetros por função e conectores complexos e, mesmo assim, manter a integridade da estrutura a ser protegida e ter fácil manutenção se tudo que está relacionado à parte protegida encontra-se no mesmo local. 

\end{homeworkProblem}


\begin{homeworkProblem}

    Nem toda coesão funcional é lógica. Certo, errado, tipo assim (depende), justifique.

    \vspace{\baselineskip}
    \textbf{Solução}

    Errado. Como a coesão funcional implica que os elementos interdependem em torno de uma funcionalidade podemos afirmar que também interdependem em torno do conceito lógico associado a ela.

\end{homeworkProblem}


\begin{homeworkProblem}

    É possível existir mais de um módulo de definição para um módulo de implementação? Porque alguém faria isso?

    \vspace{\baselineskip}
    \textbf{Solução}

	Não.

\end{homeworkProblem}


\begin{homeworkProblem}

    Explique se existe diferença entre requisito inverso e restrição. 

    \vspace{\baselineskip}
    \textbf{Solução}

	Existe sim diferença entre requisito inverso e restrição. Requisito inverso é aquilo que não será feito. Por exemplo, "Não implementaremos o login porque não foi pedido". Rrestrições são regras que restringem as alternativas de solução de um problema. Por exemplo, "A aplicação deve ser redigida em C++".

\end{homeworkProblem}


\end{document}
