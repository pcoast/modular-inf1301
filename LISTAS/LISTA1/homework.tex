\documentclass[]{article}
\usepackage[brazilian]{babel}
\usepackage[utf8]{inputenc}
\usepackage[T1]{fontenc}
\usepackage{fancyhdr}
\usepackage{extramarks}
\usepackage{amsmath}
\usepackage{amsthm}
\usepackage{amsfonts}
\usepackage{tikz}
\usepackage[plain]{algorithm}
\usepackage{algpseudocode}
\usepackage{listings}

\usepackage{xcolor}
\usepackage{listings}

\definecolor{mGreen}{rgb}{0,0.6,0}
\definecolor{mGray}{rgb}{0.5,0.5,0.5}
\definecolor{mPurple}{rgb}{0.58,0,0.82}
\definecolor{backgroundColour}{rgb}{0.95,0.95,0.92}

\lstdefinestyle{CStyle}{
    backgroundcolor=\color{backgroundColour},   
    commentstyle=\color{mGreen},
    keywordstyle=\color{magenta},
    numberstyle=\tiny\color{mGray},
    stringstyle=\color{mPurple},
    basicstyle=\footnotesize,
    breakatwhitespace=false,         
    breaklines=true,                 
    captionpos=b,                    
    keepspaces=true,                 
    numbers=left,                    
    numbersep=5pt,                  
    showspaces=false,                
    showstringspaces=false,
    showtabs=false,                  
    tabsize=2,
    language=C
}

\usetikzlibrary{automata,positioning}

%
% Basic Document Settings
%

\topmargin=-0.45in
\evensidemargin=0in
\oddsidemargin=0in
\textwidth=6.5in
\textheight=9.0in
\headsep=0.25in

\linespread{1.1}

\pagestyle{fancy}
\lhead{AVC, JPP, PC}
\chead{\hmwkClass\ (\hmwkClassInstructor\ \hmwkClassTime): \hmwkTitle}
\rhead{\firstxmark}
\lfoot{\lastxmark}
\cfoot{\thepage}

\renewcommand\headrulewidth{0.4pt}
\renewcommand\footrulewidth{0.4pt}

\setlength\parindent{0pt}

%
% Create Questão Sections
%

\newcommand{\enterProblemHeader}[1]{
    \nobreak\extramarks{}{Questão \arabic{#1} continued on next page\ldots}\nobreak{}
    \nobreak\extramarks{Questão \arabic{#1} (continued)}{Questão \arabic{#1} continued on next page\ldots}\nobreak{}
}

\newcommand{\exitProblemHeader}[1]{
    \nobreak\extramarks{Questão \arabic{#1} (continued)}{Questão \arabic{#1} continued on next page\ldots}\nobreak{}
    \stepcounter{#1}
    \nobreak\extramarks{Questão \arabic{#1}}{}\nobreak{}
}

\setcounter{secnumdepth}{0}
\newcounter{partCounter}
\newcounter{homeworkProblemCounter}
\setcounter{homeworkProblemCounter}{1}
\nobreak\extramarks{Questão \arabic{homeworkProblemCounter}}{}\nobreak{}

%
% Homework Questão Environment
%
% This environment takes an optional argument. When given, it will adjust the
% problem counter. This is useful for when the problems given for your
% assignment aren't sequential. See the last 3 problems of this template for an
% example.
%
\newenvironment{homeworkProblem}[1][-1]{
    \ifnum#1>0
        \setcounter{homeworkProblemCounter}{#1}
    \fi
    \section{Questão \arabic{homeworkProblemCounter}}
    \setcounter{partCounter}{1}
    \enterProblemHeader{homeworkProblemCounter}
}{
    \exitProblemHeader{homeworkProblemCounter}
}

%
% Homework Details
%   - Title
%   - Due date
%   - Class
%   - Section/Time
%   - Instructor
%   - Author
%

\newcommand{\hmwkTitle}{Lista 1}
\newcommand{\hmwkDueDate}{16 de outubro de 2019}
\newcommand{\hmwkClass}{Programação Modular}
\newcommand{\hmwkClassTime}{}
\newcommand{\hmwkClassInstructor}{Professor Flavio Bevilacqua}

% \setlength{\textfloatsep}{1in}
\newcommand{\hmwkAuthorName}{

\begin{tabular}{c}\textbf{Antônio Vasconcellos Chaves} \\ Engenharia da Computaçāo \\ Pontifícia Universidade Católica \\ do Rio de Janeiro \\ Rio de Janeiro, RJ 22451-900 \\ antoniovasconcelloschaves@gmail.com \end{tabular}\and

\begin{tabular}{c}\textbf{João Pedro Paiva} \\ Ciência da Computaçāo \\ Pontifícia Universidade Católica \\ do Rio de Janeiro \\ Rio de Janeiro, RJ 22451-900 \\ joaopedrordepaiva@gmail.com \end{tabular}\and

\\[\baselineskip]\begin{tabular}{c}\textbf{Pedro Moreira Costa} \\ Engenharia da Computaçāo \\ Pontifícia Universidade Católica \\ do Rio de Janeiro \\ Rio de Janeiro, RJ 22451-900 \\ pedromoreiramcosta@gmail.com \end{tabular}

}

%
% Title Page
%

\title{
    \vspace{2in}
    \textmd{\textbf{\hmwkClass:\ \hmwkTitle}}\\
    \normalsize\vspace{0.1in}\small{Entrega\ no dia\ \hmwkDueDate\ às 17h}\\
    \vspace{0.1in}\large{\textit{\hmwkClassInstructor\ \hmwkClassTime}}
    \vspace{1.25in}
}

\author{\hmwkAuthorName}
\date{}

\renewcommand{\part}[1]{\textbf{\large Parte \Alph{partCounter}}\stepcounter{partCounter}\\}

%
% Various Helper Commands
%

% Useful for algorithms
\newcommand{\alg}[1]{\textsc{\bfseries \footnotesize #1}}

% For derivatives
\newcommand{\deriv}[1]{\frac{\mathrm{d}}{\mathrm{d}x} (#1)}

% For partial derivatives
\newcommand{\pderiv}[2]{\frac{\partial}{\partial #1} (#2)}

% Integral dx
\newcommand{\dx}{\mathrm{d}x}

% Alias for the Solution section header
\newcommand{\solution}{\textbf{\large Solução}}

% Probability commands: Expectation, Variance, Covariance, Bias
\newcommand{\E}{\mathrm{E}}
\newcommand{\Var}{\mathrm{Var}}
\newcommand{\Cov}{\mathrm{Cov}}
\newcommand{\Bias}{\mathrm{Bias}}

\begin{document}

\maketitle

\pagebreak

\begin{homeworkProblem}

    Explique com um exemplo o conceito de callback.

    \vspace{\baselineskip}
    \textbf{Solução}

    Callback é o termo usado para quando ao ter os dados requisitados, enviados pelo cliente, ainda há requisitos não preenchidos que não são obrigatórios. Por exemplo, em um cadastro, o servidor precisa das informações (nome, sobrenome e CPF) do cliente. O cliente esquece de preencher o CPF. Então o servidor volta/gera para interface com um aviso de que aquele campo é de preenchimento obrigatório.

\end{homeworkProblem}

\vspace{\baselineskip}

\begin{homeworkProblem}

    Apresente um requisito funcional bem formulado, derivado de um requisito não funcional, diferente do exemplo de login visto em aula.

    \vspace{\baselineskip}
    \textbf{Solução}

    A necessidade que a solução seja descentralizada, por exemplo. Pois torna necessária a implementação de módulos para estabelecer a conexão entre os nós do sistema distribuído.

\end{homeworkProblem}

\vspace{\baselineskip}

\begin{homeworkProblem}

    Um bom acoplamento resulta em um bom encapsulamento. Certo, errado, tipo assim, justifique.

    \vspace{\baselineskip}
    \textbf{Solução}

    Errado. Ambos estão relacionados mas não têm influência um sobre o outro. O acoplamento pode ser feito com diversos parâmetros por função e conectores complexos e, mesmo assim, manter a integridade da estrutura a ser protegida e ter fácil manutenção se tudo que está relacionado à parte protegida encontra-se no mesmo local.

\end{homeworkProblem}

\vspace{\baselineskip}

\begin{homeworkProblem}

    Nem toda coesão funcional é lógica. Certo, errado, tipo assim (depende), justifique.

    \vspace{\baselineskip}
    \textbf{Solução}

    Errado. Como a coesão funcional implica que os elementos interdependem em torno de uma funcionalidade podemos afirmar que também interdependem em torno do conceito lógico associado a ela.

\end{homeworkProblem}

\vspace{\baselineskip}

\begin{homeworkProblem}

    É possível existir mais de um módulo de definição para um módulo de implementação? Porque alguém faria isso?

    \vspace{\baselineskip}
    \textbf{Solução}

    Não.

\end{homeworkProblem}

\vspace{\baselineskip}

\begin{homeworkProblem}

    Explique se existe diferença entre requisito inverso e restrição.

    \vspace{\baselineskip}
    \textbf{Solução}

    Existe sim diferença entre requisito inverso e restrição. Requisito inverso é aquilo que não será feito. Por exemplo, "Não implementaremos o login porque não foi pedido". Rrestrições são regras que restringem as alternativas de solução de um problema. Por exemplo, "A aplicação deve ser redigida em C++".

\end{homeworkProblem}

\vspace{\baselineskip}

\begin{homeworkProblem}

    \vspace{\baselineskip}
    \textbf{Solução}


\end{homeworkProblem}

\vspace{\baselineskip}

\begin{homeworkProblem}

    \vspace{\baselineskip}
    \textbf{Solução}


\end{homeworkProblem}

\vspace{\baselineskip}

\begin{homeworkProblem}

    \vspace{\baselineskip}
    \textbf{Solução}


\end{homeworkProblem}

\vspace{\baselineskip}

\begin{homeworkProblem}

    \vspace{\baselineskip}
    \textbf{Solução}


\end{homeworkProblem}

\vspace{\baselineskip}

\begin{homeworkProblem}

    Um programa deve ser capaz de ler um documento de texto e criar um índice remissivo para cada substantivo encontrado. No índice remissivo é apresentada a lista de páginas e que a palavra é encontrada. Elabore a arquitetura modularizada deste programa (tal como foi vista na disciplina) considerando a criação de um tipo abstrato de dados para a estrutura principal a ser acoplada na aplicação. Neste tipo abstrato de dados deve ser utilizada a estrutura Lista Duplamente Encadeada com Cabeça.

    \vspace{\baselineskip}
    \textbf{Solução}

\end{homeworkProblem}

\vspace{\baselineskip}

\begin{homeworkProblem}

    Uma estrutura de chamadas de funções pode ser simultaneamente recursiva direta, indireta e gerar dependência circular entre módulos. Comente essa afirmação.

    \vspace{\baselineskip}
    \textbf{Solução}


\end{homeworkProblem}

\vspace{\baselineskip}

\begin{homeworkProblem}

    Dê um exemplo de função morta existente nos seus trabalhos. Justifique sua resposta.

    \vspace{\baselineskip}
    \textbf{Solução}


\end{homeworkProblem}

\vspace{\baselineskip}

\begin{homeworkProblem}

    Dado o seguinte requisito funcional: "Após o término da entrada das notas dos alunos na funcionalidade 'Apresentar Situação Final', a Média Final é calculada e apresentada na tela.". Avalie a necessidade da criação de um requisito inverso. Caso seja necessário, apresente este requisito e explique sua necessidade. Caso não seja necessário, explique a razão de não especificar este requisito.

    \vspace{\baselineskip}
    \textbf{Solução}

    Requisitos inversos são incluídos, geralmente, para eliminar ambiguidades. No requisito apresentado, não etá claro qual Média Final deve ser calculada. Pode ser a Média Final de cada aluno ou de todos os alunos juntos. Também não está claro como a média deve ser claculada. Um requisito que poderia ser adicionado para esclarecer é: "esta aplicação não clacula a média de todos os alunos juntos, a média de cada aluno será calculada por (G1 + G2 + 2*T1)/4".

\end{homeworkProblem}

\begin{homeworkProblem}

    Especifique as asssertivas de entrada e saída de uma função que é armazenada no cabeça de uma lista para ser utilizada na exclusão do conteúdo apontado por um nó.
    
    \vspace{\baselineskip}
    \textbf{Solução}


\end{homeworkProblem}

\begin{homeworkProblem}

    A validação de requisitos é feita pela equipe técnica junto com o cliente. Certo/Errado. Justifique.
    
    \vspace{\baselineskip}
    \textbf{Solução}

    Certo. Após elicitar do cliente aquilo que ele deseja, documentar a elicitação e verificar se aquilo que o cliente deseja é factível, o Analista de Requisitos valida os requisitos documentados com o cliente.


\end{homeworkProblem}

\begin{homeworkProblem}

    Explique o que um gerente de configuração de software faz em um processo de desenvolvimento.
    
    \vspace{\baselineskip}
    \textbf{Solução}
    
    O gerente de configuração de software maneja baselines e ambientes de configuração.


\end{homeworkProblem}

\begin{homeworkProblem}

    "O relatório deve ter no máximo dez linhas por página". Requisito funcional, não funcional, restrição ou hipótese? Justifique.
    
    \vspace{\baselineskip}
    \textbf{Solução}

    Requisito não funcional. Pois descreve algo que deve ser feito, mas não está relacionado a uma funcionalidade específica.

\end{homeworkProblem}

\begin{homeworkProblem}

    Mostre através de um exemplo como é possível com um único módulo de definição gerar interfaces personalizadas para cada módulo cliente.
    
    \vspace{\baselineskip}
    \textbf{Solução}

    O seguinte módulo de definição:

    \lstinputlisting[style=CStyle]{M1.h}

    gera \lstinputlisting[style=CStyle]{resultado1.c} com o seguinte módulo cliente:

    \lstinputlisting[style=CStyle]{MS.c}

    e gera \lstinputlisting[style=CStyle]{resultado2.c} com o seguinte módulo cliente:


    \lstinputlisting[style=CStyle]{MC.c}


\end{homeworkProblem}

\begin{homeworkProblem}

    Um valor somente declarado pode estar também definido? Certo/errado. Justifique.
    
    \vspace{\baselineskip}
    \textbf{Solução}


\end{homeworkProblem}

\begin{homeworkProblem}

    É possível declarar sem definir? Porque alguém faria isso?
    
    \vspace{\baselineskip}
    \textbf{Solução}


\end{homeworkProblem}

\begin{homeworkProblem}

    O encapsulamento sempre protege os dados. Certo/errado. Justifique.
    
    \vspace{\baselineskip}
    \textbf{Solução}

    Errado. Por mais que o encapsulamento tenha sido perfeito, ainda pode-se acessar o espaço de memória de algum dado e mexer nele, já que C é uma linguagem de baixo nível.

\end{homeworkProblem}


\end{document}
