\documentclass[a4paper,12pt,oneside]{book}
%
\input{./config/packages}
\input{./config/macros}

% SRS cover page
\title{
  \begin{flushright}
  \Huge{MODELO FÍSICO ESTRUTURAL,}\\
  ASSERTIVAS\\
  e\\
  EXEMPLO FÍSICO\\
  da aplicação\\
  LABIRINTO\\
  ~\\
  \LARGE{Versão 2.0}\\
  ~\\
  INF1301 - Programação Modular\\ DI/PUC-Rio
  \end{flushright}
}

% Author
\author{Antônio Chaves - AVC\\João Pedro Paiva - JPP\\Pedro Costa - PC}
\date{\today}

\begin{document}

\frontmatter
\maketitle

\tableofcontents

\chapter*{\centering Histórico de Revisões}

\begin{center}
    \begin{tabular}{|c|c|c|c|}
        \hline
        Versão & Data       & Autor & Observações                                      \\
        \hline
        1.0    & 08/10/2019 & JPP   & Versão do Trab4                                  \\
        \hline
    \end{tabular}
\end{center}


\mainmatter

\chapter{Modelo Físico Estrutural}

\begin{center}

    \includegraphics[width=1\textwidth,height=27\baselineskip]{fisico.png}

\end{center}

\chapter{Assertivas Estruturais}\setlength{\parskip}{1\baselineskip}

    Se o pNoCorr->pNoDireita diferente de NULL, então pNoCorr->pNoDireita->pNoEsquerda = pNoCorr;

    Se o pNoCorr->pNoEsquerda diferente de NULL, então pNoCorr->pNoEsquerda->pNoDireita = pNoCorr;

    Se o pNoCorr->pNoCima diferente de NULL, então pNoCorr->pNoCima->pNoBaixo = pNoCorr;

    Se o pNoCorr->pNoBaixo diferente de NULL, então pNoCorr->pNoBaixo->pNoCima = pNoCorr;
    \setlength{\parskip}{1pt}
\chapter{Exemplo Físico}

\begin{center}
    
    \includegraphics[width=\textwidth,height=27\baselineskip]{exemplo.png}

\end{center}

% \input{./chapter_preface/chapter_preface}
% \chapter{Introdução}

\section{Objetivo}
% $<$Identify the product whose software requirements are specified in this
% document, including the revision or release number. Describe the scope of the
% product that is covered by this SRS, particularly if this SRS describes only
% part of the system or a single subsystem.$>$
Este documento especifica os requisitos da aplicação Labirinto.

% \section{Document Conventions}
% $<$Describe any standards or typographical conventions that were followed when
% writing this SRS, such as fonts or highlighting that have special significance.
% For example, state whether priorities  for higher-level requirements are assumed
% to be inherited by detailed requirements, or whether every requirement statement
% is to have its own priority.$>$

\section{Público Alvo}
% $<$Describe the different types of reader that the document is intended for,
% such as developers, project managers, marketing staff, users, testers, and documentation writers. Describe what the rest of this SRS contains and how it is
% organized. Suggest a sequence for reading the document, beginning with the
% overview sections and proceeding through the sections that are most pertinent to
% each reader type.$>$
Este documento fornece aos projetistas e desenvolvedores as informações necessárias para o projeto e implementação, assim como para a realização dos testes e homologação do sistema.

\section{Escopo do Projeto}
% $<$Provide a short description of the software being specified and its purpose,
% including relevant benefits, objectives, and goals. Relate the software to
% corporate goals or business strategies. If a separate vision and scope document
% is available, refer to it rather than duplicating its contents here.$>$
A aplicação Labirinto permite construir um labirinto e disponibiliza um resolvedor para o mesmo.

% \section{Referências}
% $<$List any other documents or Web addresses to which this SRS refers. These may
% include user interface style guides, contracts, standards, system requirements
% specifications, use case documents, or a vision and scope document. Provide
% enough information so that the reader could access a copy of each reference,
% including title, author, version number, date, and source or location.$>$

% \chapter{Descrição Geral}

\section{Perspectiva do Produto}
% $<$Describe the context and origin of the product being specified in this SRS.
% For example, state whether this product is a follow-on member of a product
% family, a replacement for certain existing systems, or a new, self-contained
% product. If the SRS defines a component of a larger system, relate the
% requirements of the larger system to the functionality of this software and
% identify interfaces between the two. A simple diagram that shows the major
% components of the overall system, subsystem interconnections, and external
% interfaces can be helpful.$>$
A aplicação Labirinto é completamente autocontida, não é um componente de outro sistema.

\section{Funções do Projeto}
% $<$Summarize the major functions the product must perform or must let the user
% perform. Details will be provided in Section 3, so only a high level summary
% (such as a bullet list) is needed here. Organize the functions to make them
% understandable to any reader of the SRS. A picture of the major groups of
% related requirements and how they relate, such as a top level data flow diagram
% or object class diagram, is often effective.$>$
Será possível construir manualmente um labirinto, desenhando com caracteres em um arquivo txt, e uma função deverá imprimir o caminho da entrada até a saída.

% \section{User Classes and Characteristics}
% $<$Identify the various user classes that you anticipate will use this product.
% User classes may be differentiated based on frequency of use, subset of product
% functions used, technical expertise, security or privilege levels, educational
% level, or experience. Describe the pertinent characteristics of each user class.
% Certain requirements may pertain only to certain user classes. Distinguish the
% most important user classes for this product from those who are less important
% to satisfy.$>$

\section{Ambiente de Operação}
% $<$Describe the environment in which the software will operate, including the
% hardware platform, operating system and versions, and any other software
% components or applications with which it must peacefully coexist.$>$
A aplicação deve operar em computadores 64-bit com OS Windows 7 ou superior.

\section{Restrições de Design e Implementação}
% $<$Describe any items or issues that will limit the options available to the
% developers. These might include: corporate or regulatory policies; hardware
% limitations (timing requirements, memory requirements); interfaces to other
% applications; specific technologies, tools, and databases to be used; parallel
% operations; language requirements; communications protocols; security
% considerations; design conventions or programming standards (for example, if the
% customer’s organization will be responsible for maintaining the delivered
% software).$>$
A aplicação Labirinto deve ser desenvolvida por completo utilizando a linguagem de programação C. Será obrigatório que a arquitetura da aplicação tenha pelo menos dois tipos abstratos de dados, o módulo de Labirinto e o módulo de Matriz, além de um módulo centralizador principal. O módulo de Labirinto deve chamar o módulo Matriz. É proibida a utilização de listas encadeadas e qualquer estrutura estática (por exemplo: vetores estáticos), ou que aloca um espaço de memória contíguo para armazenar a estrutura do labirinto. Todos os programas devem estar em conformidade com os padrões dos apêndices de 1 a 10 do livro Programação Modular escrito por Arnt von Staa. Em particular, os módulos e funções devem estar devidamente especificados.

% \section{User Documentation}
% $<$List the user documentation components (such as user manuals, on-line help,
% and tutorials) that will be delivered along with the software. Identify any
% known user documentation delivery formats or standards.$>$

\section{Suposições e Dependências}
% $<$List any assumed factors (as opposed to known facts) that could affect the
% requirements stated in the SRS. These could include third-party or commercial
% components that you plan to use, issues around the development or operating
% environment, or constraints. The project could be affected if these assumptions
% are incorrect, are not shared, or change. Also identify any dependencies the
% project has on external factors, such as software components that you intend to
% reuse from another project, unless they are already documented elsewhere (for
% example, in the vision and scope document or the project plan).$>$
A versão final da aplicação Labirinto, a ser completada no Trabalho 3, rodará fora do arcabouço
com uma interface com o usuário, ou seja, terá um módulo principal.

% \input{./chapter_externalInterfaceRequirements/chapter_externalInterfaceRequirements}
% \chapter{Funcionalidades do Sistema}
% $<$This template illustrates organizing the functional requirements for the
% product by system features, the major services provided by the product. You may
% prefer to organize this section by use case, mode of operation, user class,
% object class, functional hierarchy, or combinations of these, whatever makes the
% most logical sense for your product.$>$

\section{Funcionalidade 1}
% $<$Don’t really say “System Feature 1.” State the feature name in just a few
% words.$>$

Construção de um labirinto.

\subsection{Descrição e Prioridade}
% $<$Provide a short description of the feature and indicate whether it is of
% High, Medium, or Low priority. You could also include specific priority
% component ratings, such as benefit, penalty, cost, and risk (each rated on a
% relative scale from a low of 1 to a high of 9).$>$

Funcionalidade que constrói o labirinto a ser resolvido. Prioridade alta. Funcionalidade essencial para o funcionamento da aplicação.

\subsection{Sequências de Estímulo e Resposta}
% $<$List the sequences of user actions and system responses that stimulate the
% behavior defined for this feature. These will correspond to the dialog elements
% associated with use cases.$>$
O usuário contrói o labirinto manulamente, escrevendo em um arquivo txt. Ao executar a aplicação Labirinto, o labirinto salvo no txt é armazenado, pelo módulo Labirinto, numa matriz criada pelo módulo Matriz.

\subsection{Requisitos Funcionais}
% $<$Itemize the detailed functional requirements associated with this feature.
% These are the software capabilities that must be present in order for the user
% to carry out the services provided by the feature, or to execute the use case.
% Include how the product should respond to anticipated error conditions or
% invalid inputs. Requirements should be concise, complete, unambiguous,
% verifiable, and necessary. Use “TBD” as a placeholder to indicate when necessary
% information is not yet available.$>$

% $<$Each requirement should be uniquely identified with a sequence number or a
% meaningful tag of some kind.$>$

\begin{enumerate}
    \item O arquivo de texto criado pelo usuário deve ser lido durante a execução desta funcionalidade.
    \item Uma matriz de dimensões adequadas, inferidas a partir do arquivo de texto, é criada.
    \item A funcionalidade de criação do labirinto deve armazenar na matriz dados que possibilitem navegar dentro do labirinto, sinalizando os caminhos possíveis.
\end{enumerate}

\section{Funcionalidade 2}
% $<$Don’t really say “System Feature 1.” State the feature name in just a few
% words.$>$

Resolve labirinto.

\subsection{Descrição e Prioridade}
% $<$Provide a short description of the feature and indicate whether it is of
% High, Medium, or Low priority. You could also include specific priority
% component ratings, such as benefit, penalty, cost, and risk (each rated on a
% relative scale from a low of 1 to a high of 9).$>$

Funcionalidade que resolve o labirinto construído. Prioridade alta. Funcionalidade essencial para o funcionamento da aplicação.

\subsection{Sequências de Estímulo e Resposta}
% $<$List the sequences of user actions and system responses that stimulate the
% behavior defined for this feature. These will correspond to the dialog elements
% associated with use cases.$>$

Após o usuário executar a aplicação Labirinto e o labirinto ser criado pela aplicação, a aplicação encontra e demarca um caminho que parte do início e chega no final do labirinto.

\subsection{Requisitos Funcionais}
% % $<$Itemize the detailed functional requirements associated with this feature.
% % These are the software capabilities that must be present in order for the user
% % to carry out the services provided by the feature, or to execute the use case.
% % Include how the product should respond to anticipated error conditions or
% % invalid inputs. Requirements should be concise, complete, unambiguous,
% % verifiable, and necessary. Use “TBD” as a placeholder to indicate when necessary
% % information is not yet available.$>$

% % $<$Each requirement should be uniquely identified with a sequence number or a
% % meaningful tag of some kind.$>$

\begin{enumerate}
    \item A funcionalidade de resolução do labirinto utiliza aprendizado de máquina para encontrar a solução.
    \item Caso o labirinto não possua solução, a funcionalidade de resolução do labirinto deve deixar isto evidente.
    \item A solução do labirinto é sinalizada na própria estrutura do labirinto.
\end{enumerate}

\section{Funcionalidade 3}
% $<$Don’t really say “System Feature 1.” State the feature name in just a few
% words.$>$

Imprime solução labirinto.

\subsection{Descrição e Prioridade}
% $<$Provide a short description of the feature and indicate whether it is of
% High, Medium, or Low priority. You could also include specific priority
% component ratings, such as benefit, penalty, cost, and risk (each rated on a
% relative scale from a low of 1 to a high of 9).$>$

Funcionalidade que imprime o labirinto resolvido com o caminho do início até o fim demarcado para o usuário. Prioridade alta. Funcionalidade essencial para o funcionamento da aplicação.

\subsection{Sequências de Estímulo e Resposta}
% $<$List the sequences of user actions and system responses that stimulate the
% behavior defined for this feature. These will correspond to the dialog elements
% associated with use cases.$>$
Após o usuário executar a aplicação Labirinto, o labirinto ser criado pela aplicação e a aplicação resolver o labirinto, o labirinto será impresso para o usuário com a sua resolução demarcada.

\subsection{Requisitos Funcionais}
% $<$Itemize the detailed functional requirements associated with this feature.
% These are the software capabilities that must be present in order for the user
% to carry out the services provided by the feature, or to execute the use case.
% Include how the product should respond to anticipated error conditions or
% invalid inputs. Requirements should be concise, complete, unambiguous,
% verifiable, and necessary. Use “TBD” as a placeholder to indicate when necessary
% information is not yet available.$>$

% $<$Each requirement should be uniquely identified with a sequence number or a
% meaningful tag of some kind.$>$

\begin{enumerate}
    \item A impressão deve utilizar caracteres ASCII.
    \item A impressão aparecerá na tela para o usuário.
\end{enumerate}


% \input{./chapter_otherNonFunctionalRequirements/chapter_otherNonFunctionalRequirements}
% \input{./chapter_otherRequirements/chapter_otherRequirements}

\end{document}
